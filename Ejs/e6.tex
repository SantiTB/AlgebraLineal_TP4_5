\item Hallar el m.c.d. entre los siguientes polinomios de $\R[x]$.
    \begin{enumerate}
        \item $p(x)=x^5-4x^4-3x+1$ y $q(x)=3x^2+2x+1$.
            \begin{mdframed}[style=s]
                El máximo común divisor es el generador mónico $d\in\K[x]$ del ideal\[M=\{(x^5-4x^4-3x+1)f+(3x^2+2x+1)g\quad f,g\in\K[x]\}\]
                Por el \textbf{Corolario 3.19}, sabemos que $d$ divide a $p(x)$ y $q(x)$. Al factorizar los polinomios, estos no comparten ninguna raíz, por ende\[mcd(p,q)=1\]
            \end{mdframed}
        \item $p(x)=x^4-2x^3+1$ y $q(x)=x^2-x+2$.
            \begin{mdframed}[style=s]
                En este caso sucede lo mismo.
            \end{mdframed}
        \item $p(x)=2x^3-4x^2+x-1$ y $q(x)=x^3-x^2+2x$.
            \begin{mdframed}[style=s]
                En este caso también.
            \end{mdframed}
    \end{enumerate}