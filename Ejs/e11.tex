\item Sean $T\in L(\R^3)$ dada por $T(x,y,z)=(-x,-z,2y)$ y $p(x)=x^2+3\in\R[x]$.
    \begin{enumerate}
        \item Hallar $p(T)$ y calcular $(pT)(1,0,-1)$.
            \begin{mdframed}[style=s]
                \begin{center}
                    $[T]_\E=\begin{pmatrix}
                        -1&0&0\\0&0&-1\\0&2&0
                    \end{pmatrix}\to p([T_\E])=\begin{pmatrix}
                        4&0&0\\0&1&0\\0&0&1
                    \end{pmatrix}\to (pT)(1,0,-1)=\begin{pmatrix}
                        4\\0\\-1
                    \end{pmatrix}$
                \end{center}
            \end{mdframed}
        \item Probar que $p(T)\in L(\R^3)$.
            \begin{mdframed}[style=s]
                $p(T)(x,y,z)=(4x,y,z)$. A partir de acá es fácil comprobar que $p(T)\in L(\R^3)$
            \end{mdframed}
        \item Escribir la representación matricial de $p(T)$ en la base canónica de $\R^3$.
            \begin{mdframed}[style=s]
                \begin{center}
                    $[p(T)]_E=\begin{pmatrix}
                        4&0&0\\0&1&0\\0&0&1
                    \end{pmatrix}$
                \end{center}
            \end{mdframed}
        \item Probar que $[p(T)]_\E=p([T]_\E)$.
            \begin{mdframed}[style=s]
                Ya se mostró.
            \end{mdframed}
    \end{enumerate}