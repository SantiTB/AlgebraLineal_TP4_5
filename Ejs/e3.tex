\item \textit{Raíces de polinomios de grado dos.} Sea $p(x)=x^2+bx+c\in\R[x]$. Probar que 
\[p(x)=(x-\lambda_1)(x-\lambda_2)\quad\text{para}\quad\lambda_1,\lambda_2\in\R,\]
    si y sólo si $b^2\geq 4c$.
    \begin{mdframed}[style=s]
        \begin{itemize}
            \item $\Rightarrow$\\
                Las raíces de $p$ son: $\lambda_{1,2}=\frac{-b\pm\sqrt{b^2-4c}}{2}$. Para que $\lambda_{1,2}\in\R\to b^2\geq4c$
            \item $\Leftarrow$\\
                Si $b^2\geq4c\to$ llamando $\lambda_1$ y $\lambda_2$ a las raíces de $p$ tenemos que $p(x)=(x-\lambda_1)(x-\lambda_2)\quad \lambda_1,\lambda_2\in\R$
        \end{itemize}
    \end{mdframed}